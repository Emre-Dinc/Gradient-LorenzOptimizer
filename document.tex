\documentclass[11pt]{article}
\usepackage{amsmath, amssymb}
\usepackage{geometry}
\geometry{margin=1in}
\usepackage{titlesec}
\usepackage{enumitem}
\titleformat{\section}{\normalfont\Large\bfseries}{}{0em}{}

\title{Component-Level Structural Review of the Global Gradient-Guided Chaos Optimizer}
\date{}
\begin{document}
\maketitle

\section*{1. Lorenz Chaos Engine}
\textbf{Purpose:} Introduce structured deterministic exploration via a low-dimensional chaotic system.

\textbf{Structure:}
\begin{itemize}[leftmargin=1.5em]
  \item 3D Lorenz system with standard parameters ($\sigma=10$, $\rho=28$, $\beta=\frac{8}{3}$)
  \item Temporal integration via Euler ($\Delta t = 0.01$)
  \item Output: 3D trajectory $(x, y, z)$
\end{itemize}

\textbf{Function:}
\begin{itemize}[leftmargin=1.5em]
  \item Drives self-evolving chaotic exploration
  \item Prevents premature stagnation
  \item Maintains global exploration pressure
\end{itemize}

\textbf{Why this method:} Deterministic chaos enables rich dynamics without stochastic noise. Euler is simple and fast.

\section*{2. Chaos Projection into High Dimensions}
\textbf{Purpose:} Map the 3D Lorenz trajectory into a $D$-dimensional search vector.

\textbf{Structure:}
\[
\text{chaos\_flow}_i = \sin(f_i \cdot t + x + y + z), \quad f_i = \text{base frequency} \cdot (i+1)
\]

\textbf{Function:} Provides decorrelated perturbations across dimensions, enabling high-dimensional exploration.

\textbf{Planned Improvements:} Replace sine mapping with attractor coupling or OGY control.

\section*{3. Gradient Guidance Module}
\textbf{Purpose:} Anchor the optimizer to local energy gradients.

\textbf{Structure:}
\[
\hat{\nabla}f(x) = \frac{\nabla f(x)}{\|\nabla f(x)\| + \epsilon}
\]

\textbf{Function:} Guides the optimizer along descent directions.

\textbf{Justification:} Ensures convergence pressure while maintaining chaotic flow.

\section*{4. Directional Interpolation (Fusion Layer)}
\textbf{Purpose:} Fuse chaotic and gradient flows.

\textbf{Structure:}
\[
\vec{d}_t = (1 - \alpha) \cdot \vec{c}_t + \alpha \cdot \hat{\nabla}f(x)
\]

\textbf{Function:} Dynamically balances exploration ($\vec{c}_t$) and exploitation ($\hat{\nabla}f(x)$).

\textbf{Justification:} Enables adaptive shift between chaotic search and gradient convergence.

\section*{5. Momentum Module}
\textbf{Purpose:} Retain directional memory.

\textbf{Structure:}
\[
  v_t = \beta \cdot v_{t-1} + \eta \cdot \vec{d}_t
\]

\textbf{Function:} Smooths optimizer trajectory, stabilizes valley navigation.

\textbf{Justification:} Necessary for anisotropic landscapes (e.g., Rosenbrock).

\section*{6. Adaptive Control Subsystem}
\textbf{Purpose:} Dynamically modulate key control parameters.

\textbf{Structure:}
\begin{itemize}[leftmargin=1.5em]
  \item Triggered every $N$ iterations
  \item Adjusts $\alpha$, chaos amplitude, and step size
\end{itemize}

\textbf{Function:} Phase adaptation (explore $\rightarrow$ converge)

\textbf{Justification:} Eliminates need for manual tuning or predefined schedules.

\section*{7. Future Integration: OGY Control}
\textbf{Purpose:} Replace global chaotic projection with local chaos control.

\textbf{Structure:}
\begin{itemize}[leftmargin=1.5em]
  \item Monitor unstable orbits
  \item Apply minimal corrections near bifurcation points
\end{itemize}

\textbf{Function:} Enables gradient-free convergence with deterministic chaos.

\textbf{Justification:} Extends applicability to non-smooth or noisy functions.

\section*{Current Limitations}
\begin{enumerate}[leftmargin=1.5em]
  \item \textbf{Projection Bottleneck:} Sine-based mapping may distort attractor topology.
  \item \textbf{Multimodal Performance:} Poorer performance on highly multimodal functions.
  \item \textbf{Gradient Dependence:} Still relies on differentiability of the objective.
  \item \textbf{Stabilization Lacking:} No mechanism for periodic orbit locking.
  \item \textbf{Single Attractor Constraint:} Current use of single Lorenz system limits complexity.
\end{enumerate}

\section*{Summary}
This optimizer architecture integrates curvature-sensitive chaotic dynamics with adaptive gradient descent to build a deterministic, feedback-regulated optimizer. It is modular, extensible, and suitable for high-dimensional smooth landscapes with structured complexity.

\end{document}
